% Typeset exams with solutions in LaTeX
\documentclass{exam}
\usepackage{ragged2e}
\usepackage{graphicx}
\usepackage[german]{babel}

%\chead{Quiz}
\title{Quiz zu Betriebsrat und JAV }
\author{Thanh-Viet Nguyen}
\date{(CC BY-SA 4.0)}

\begin{document}
\maketitle
\begin{figure} [h]
	\centering 
	\includegraphics[width=0.1\linewidth]{img/88x31.png}
\end{figure}
\begin{center}	
	\begin{questions}
		%1
		\question Die allgemeine Regelung der Beziehungen zwischen Arbeitgebern und Arbeitnehmern ist festgelegt im/in der
		\begin{checkboxes}
			\choice Betriebsverfassungsgesetz 
			\choice Handelsgesetzbuch 
			\choice Mitbestimmungsgesetz  
			\choice  Wirtschafts- und Sozialordnung
		\end{checkboxes}
		\newpage
		\printanswers
		allgemeine Regelung der Beziehungen zwischen Arbeitgebern und Arbeitnehmern ist festgelegt im/in der
		\begin{checkboxes}
			\CorrectChoice Betriebsverfassungsgesetz 
			\choice Handelsgesetzbuch 
			\choice Mitbestimmungsgesetz  
			\choice Wirtschafts- und Sozialordnung
		\end{checkboxes}
		
		%2
		\question  Das Betriebsverfassungsgesetz regelt u. a. die
		\begin{checkboxes}
			\choice Arbeitzeit
			\choice Betriebsordnung 
			\choice This is the correct answer.
			\choice Kündigungsfristen
			\choice Leistungsbeurteilungsmerkmale
			\choice soziale Mitbestimmung der Arbeitnehmer im Betrieb
		\end{checkboxes}
		\newpage
		\printanswers 
		Das Betriebsverfassungsgesetz regelt u. a. die
		\begin{checkboxes}
			\choice Arbeitzeit
			\choice Betriebsordnung 
			\choice Kündigungsfristen
			\choice Leistungsbeurteilungsmerkmale
			\CorrectChoice soziale Mitbestimmung der Arbeitnehmer im Betrieb
		\end{checkboxes}

		%3
		\question Nach dem Betriebsverfassungsgesetz ist ein Betriebsrat zu wählen in Betrieben
		\begin{checkboxes}
		\choice gewerblicher Art
		\choice mit mindestens 5
		\choice mit mindestens 10
		\choice mit mindestens 50 wahlberechtigten Arbeitnehmern
		\end{checkboxes}
		\newpage
		\printanswers
		Nach dem Betriebsverfassungsgesetz ist ein Betriebsrat zu wählen in Betrieben
		\begin{checkboxes}
			\choice gewerblicher Art
			\CorrectChoice mit mindestens 5
			\choice mit mindestens 10
			\choice mit mindestens 50 wahlberechtigten Arbeitnehmern
		\end{checkboxes}
		
		%4
		\question Die Stellung des Betriebsrates ist geregelt  in der/im
		\begin{checkboxes}
			\choice 
			\choice Betriebsordnung 
			\choice Gewerbeordnung
			\choice Tarifvertragsgesetz
			\choice Betriebsverfassungsgesetz
		\end{checkboxes}
		\newpage
		\printanswers 
		Die Stellung des Betriebsrates ist geregelt  in der/im
		\begin{checkboxes}
			\choice Die Stellung des Betriebsrates ist geregelt  in der/im
			\choice Betriebsordnung 
			\choice Gewerbeordnung
			\choice Tarifvertragsgesetz
			\CorrectChoice Betriebsverfassungsgesetz
		\end{checkboxes}
		
		%5	
		\question Nach dem Betriebsverfassungsgesetz sind bei der Wahl der Betriebsjugendvertretung wahlberechtigt 
		\begin{checkboxes}
			\choice Arbeitnehmer unter 18 Jahren 
			\choice Auszubildenden
			\choice ledigen Arbeitnehmer bis 24 Jahre
			\choice Arbeitnehmer unter 21 Jahren
			\choice Arbeitnehmer zwischen 18 und 24 Jahren
		\end{checkboxes}
		\newpage
		\printanswers
		Nach dem Betriebsverfassungsgesetz sind bei der Wahl der Betriebsjugendvertretung wahlberechtigt 
		\begin{checkboxes}
			\CorrectChoice Arbeitnehmer unter 18 Jahren 
			\choice Auszubildenden
			\choice ledigen Arbeitnehmer bis 24 Jahre
			\choice Arbeitnehmer unter 21 Jahren
			\choice Arbeitnehmer zwischen 18 und 24 Jahren
		\end{checkboxes}
	
		%6	
		\question Wer vertritt die Rechte aller Arbeitnehmer im Betrieb?
		\begin{checkboxes}
			\choice Industrie- und Handelskammer
			\choice Betriebsrat
			\choice Kammer für Handelssachen
			\choice Gewerkschaften
			\choice Gewerbeaufsichtsamt
		\end{checkboxes}
		\newpage
		\printanswers
		Wer vertritt die Rechte aller Arbeitnehmer im Betrieb?
		\begin{checkboxes}
			\choice Industrie- und Handelskammer
			\CorrectChoice Betriebsrat
			\choice Kammer für Handelssachen
			\choice Gewerkschaften
			\choice Gewerbeaufsichtsamt
		\end{checkboxes}
		
		%7
		\question  Bei der Betriebsratswahl sind wahlberechtigt alle Arbeitnehmer
		\begin{checkboxes}
			\choice über 21 Jahren
			\choice des gesamten Betriebes
			\choice ab 18 Jahren
			\choice mit mindestens 6 Monaten 
		\end{checkboxes}
		\newpage
		\printanswers
		Bei der Betriebsratswahl sind wahlberechtigt alle Arbeitnehmer
		\begin{checkboxes}
			\choice über 21 Jahren
			\choice des gesamten Betriebes
			\CorrectChoice ab 18 Jahren
			\choice mit mindestens 6 Monaten 
		\end{checkboxes}
	
		%8	
		 \question Der Betriebsrat muss eine Jugendvertretung haben, wenn im Betrieb mindestens tätig sind: 
		\begin{checkboxes}
			\choice 5 Jugendliche unter 21 Jahren
			\choice 10 Jugendliche unter 21 Jahren 
			\choice 5 Jugendliche unter 21 Jahren
			\choice 5 Jugendliche unter 18 Jahren
		\end{checkboxes}
		\newpage	
		\printanswers
		Der Betriebsrat muss eine Jugendvertretung haben, wenn im Betrieb mindestens tätig sind: 
		\begin{checkboxes}
			\choice 5 Jugendliche unter 21 Jahren
			\choice 10 Jugendliche unter 21 Jahren 
			\choice 5 Jugendliche unter 21 Jahren
			\CorrectChoice 5 Jugendliche unter 18 Jahren
		\end{checkboxes}
	
	\end{questions}
\end{center}

\end{document}