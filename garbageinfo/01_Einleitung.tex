Ein Betriebsrat ist die Arbeitnehmervertretung in Betrieben, Unternehmen und Konzernen. Seine Rechte und Pflichten sind im Betriebsverfassungsgesetz geregelt. Umgangssprachlich werden auch einzelne Betriebsratsmitglieder als Betriebsräte bezeichnet. Betriebe mit Betriebsrat zahlen im Schnitt ein höheres Entgelt, die Arbeitsplätze sind sicherer und die Arbeitsbedingungen besser. Betriebsräte machen sich für die Belegschaft stark. Sie helfen bei individuellen Problemen und Konflikten am Arbeitsplatz und sie tragen zu mehr Demokratie im Betrieb bei. Geregelt ist ihre Arbeit und Mitbestimmung im Betriebsverfassungsgesetz.
\newline
Die betriebliche Mitbestimmung, wie sie durch den Betriebsrat repräsentiert wird, gibt den Arbeitnehmern die rechtliche Möglichkeit mitzureden, wenn es um betriebliche Belange geht.
\newline
Es gibt Betriebsräte auf betrieblicher Ebene, aber auch Gesamt- und Konzernbetriebsräte. In Verwaltungen und Behörden des öffentlichen Dienstes gibt es keine Betriebsräte sondern Personalräte, deren Rechte im Personalvertretungsgesetz des Bundes und der Länder geregelt sind.
\newline

