\subsection{Funktion und Arbeit des Betriebsrats}
Zwar sind Arbeitgeber aufgrund der rechtlichen Besitzverhältnisse grundsätzlich frei in ihren unternehmerischen Entscheidungen, diese sind allerdings durch die rahmengebende Gesetzgebung zur Mitbestimmung im Betrieb begrenzt. Dies dient dem Schutz der Arbeitnehmer und findet beispielsweise Anwendung in folgenden Bereichen: 
\newline
\begin{itemize}
	\item Mitbestimmung bei der Arbeitsgestaltung und Rahmenbedingungen durch Vorschlagsrecht
	\item Anspruch auf Aufklärung zur auszuübenden Tätigkeit und damit verbundener Verantwortung
	\item Einhaltung des Arbeitsschutzes und der Beurteilung von Gefährdungen
	\item Recht der Arbeitnehmer auf Einsicht in die Personalakte
	\item Recht der Vertreter der Arbeitnehmer zur Mitbestimmung bei unternehmerischen Entscheidungen, wie
	\begin{itemize}
		\item Zeiterfassung der Arbeitnehmer
		\item Kontroll- und Bewertungssysteme der Arbeitnehmer
		\item Personalplanung
		\item Sozialplan und Interessenausgleich bei unternehmerischen Umstrukturierungen
		\item Einführung von Insentives und anderen Anreizsystemen
		\item Auswahl von Mitarbeitern und deren Ausscheiden durch Kündigung
		\item Ausgestaltung von Betriebsvereinbarungen
	\end{itemize}
\end{itemize}
Die allgemeinen Aufgaben des Betriebsrats sind in Paragraf 80 Absatz 1 des Betriebsverfassungsgesetzes geregelt. Danach hat der Betriebsrat folgende allgemeine Aufgaben:
\begin{itemize}
	\item 
	Der Betriebsrat hat darüber zu wachen, dass die zugunsten der Arbeitnehmer geltenden Gesetze, Verordnungen und Unfallverhütungsvorschriften, Tarifverträge und Betriebsvereinbarungen vom Arbeitgeber eingehalten werden;
	\item
	Maßnahmen, die dem Betrieb und der Belegschaft dienen, beim Arbeitgeber zu beantragen. Die Durchsetzung der tatsächlichen Gleichstellung von Frauen und Männern zu fördern, insbesondere bei der Einstellung, Beschäftigung, Aus-, Fort- und Weiterbildung und dem beruflichen Aufstieg und die Vereinbarkeit von Familie und Erwerbstätigkeit zu fördern;
	\item 
	Anregungen von Arbeitnehmerinnen und Arbeitnehmern und der Jugend- und Auszubildendenvertretung entgegenzunehmen und, falls sie berechtigt erscheinen, durch Verhandlungen mit dem Arbeitgeber auf eine Erledigung hinzuwirken; er hat die betreffenden Arbeitnehmer über den Stand und das Ergebnis der Verhandlungen zu unterrichten;
	\item
	Die Eingliederung schwerbehinderter Menschen einschließlich der Förderung des Abschlusses von Inklusionsvereinbarungen nach § 166 des Neunten Buches Sozialgesetzbuch und sonstiger besonders schutzbedürftiger Personen zu fördern;
	\item
	Die Wahl einer Jugend- und Auszubildendenvertretung vorzubereiten und durchzuführen und mit dieser zur Förderung der Belange der in § 60 Abs. 1 genannten Arbeitnehmer eng zusammenzuarbeiten; er kann von der Jugend- und Auszubildendenvertretung Vorschläge und Stellungnahmen anfordern;
	\item
	Die Beschäftigung älterer Arbeitnehmerinnen und Arbeitnehmer im Betrieb zu fördern;
	\item
	Die Integration ausländischer Arbeitnehmer im Betrieb und das Verständnis zwischen ihnen und den deutschen Arbeitnehmern zu fördern, sowie Maßnahmen zur Bekämpfung von Rassismus und Fremdenfeindlichkeit im Betrieb zu beantragen;
	\item
	Die Beschäftigung im Betrieb zu fördern und zu sichern;
	\item
	Maßnahmen des Arbeitsschutzes und des betrieblichen Umweltschutzes zu fördern.
\end{itemize}

