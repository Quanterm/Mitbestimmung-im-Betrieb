Zwar sind Arbeitgeber aufgrund der rechtlichen Besitzverhältnisse grundsätzlich frei in ihren unternehmerischen Entscheidungen, diese sind allerdings durch die rahmengebende Gesetzgebung zur Mitbestimmung im Betrieb begrenzt. Dies dient dem Schutz der Arbeitnehmer und findet beispielsweise Anwendung in folgenden Bereichen: 
\newline
\begin{itemize}
	\item Mitbestimmung bei der Arbeitsgestaltung und Rahmenbedingungen durch Vorschlagsrecht
	\item Anspruch auf Aufklärung zur auszuübenden Tätigkeit und damit verbundener Verantwortung
	\item Einhaltung des Arbeitsschutzes und der Beurteilung von Gefährdungen
	\item Recht der Arbeitnehmer auf Einsicht in die Personalakte
	\item Recht der Vertreter der Arbeitnehmer zur Mitbestimmung bei unternehmerischen Entscheidungen, wie
	\begin{itemize}
		\item Zeiterfassung der Arbeitnehmer
		\item Kontroll- und Bewertungssysteme der Arbeitnehmer
		\item Personalplanung
		\item Sozialplan und Interessenausgleich bei unternehmerischen Umstrukturierungen
		\item Einführung von Insentives und anderen Anreizsystemen
		\item Auswahl von Mitarbeitern und deren Ausscheiden durch Kündigung
		\item Ausgestaltung von Betriebsvereinbarungen
	\end{itemize}
\end{itemize}