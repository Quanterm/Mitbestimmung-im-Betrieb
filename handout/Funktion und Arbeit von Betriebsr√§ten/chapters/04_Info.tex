\subsection*{Betriebsrätemodernisierungsgesetz}
{
	Erleichterungen nach dem Betriebsrätemodernisierungsgesetz: 
	Mit dem Betriebsrätemodernisierungsgsetz in 2021 ist die Gründung von Betriebsräten erleichtert worden, die Mitbestimmungsrechte bestehender Betriebsräte wurden erweitert. Konkret bedeutet das u.a.: Das vereinfachte Wahlverfahren zur Gründung eines Betriebsrats gilt jetzt auch in Unternehmen mit bis zu 100 Beschäftigten, statt wie früher nur bis zu einer Belegschaft von 50 Beschäftigten. Und demnach sind nur in Unternehmen mit mehr als 20 Beschäftigten zwei Unterschriften für Wahlvorschläge erforderlich. In Firmen mit weniger als 20 Beschäftigten werden gar keine Unterschriften von Beschäftigten mehr benötigt, um einen Betriebsrat zu gründen.
}
Weitere Gesetze und Regelungen: Der Betriebsrat muss für die Einhaltung von Gesetzen, Grundrechten und Arbeitsverträgen sorgen: Dazu gehören Arbeitsgesetze, Tarifverträge sowie die Arbeitsverträge, die für die Beschäftigten gelten. Der Betriebsrat hat darüber zu wachen, dass die zugunsten der Arbeitnehmer geltenden Gesetze, Verordnungen und Unfallverhütungsvorschriften, Tarifverträge und Betriebsvereinbarungen vom Arbeitgeber eingehalten werden. (Paragraf 80 Absatz 1)
\newline 
Die Rechte und Grundsätze der Zusammenarbeit von Betriebsräten mit dem Arbeitgeber sind im Betriebsverfassungsgesetz festgeschrieben. Dort sind auch die Arbeitsfelder genannt, in denen er mitbestimmen darf:
\newline 
Größe des Betriebsrats: Wenn es genug Wahlberechtigte und wählbare Beschäftigte gibt und sich genug Kandidaten finden, kann gewählt werden, auch wenn die Mehrheit noch nicht überzeugt ist. Bei bis zu 20 Wahlberechtigten wird eine Person gewählt, bei bis zu 50 drei, bei bis zu 100 sind es fünf, bei 200 sind es sieben und bei bis zu 400 Wahlberechtigten sind es neun Betriebsratsmitglieder.
