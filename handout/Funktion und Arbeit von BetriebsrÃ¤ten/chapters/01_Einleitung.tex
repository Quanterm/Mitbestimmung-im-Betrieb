Ein Betriebsrat ist die Arbeitnehmervertretung in Betrieben, Unternehmen und Konzernen. Seine Rechte und Pflichten sind im Betriebsverfassungsgesetz geregelt. Umgangssprachlich werden auch einzelne Betriebsratsmitglieder als Betriebsräte bezeichnet. 
\newline
Betriebe mit Betriebsrat zahlen im Schnitt ein höheres Entgelt, die Arbeitsplätze sind sicherer und die Arbeitsbedingungen besser. Betriebsräte machen sich für die Belegschaft stark. Sie helfen bei individuellen Problemen und Konflikten am Arbeitsplatz und sie tragen zu mehr Demokratie im Betrieb bei. Geregelt ist ihre Arbeit und Mitbestimmung im Betriebsverfassungsgesetz.
\newline
Die betriebliche Mitbestimmung, wie sie durch den Betriebsrat repräsentiert wird, gibt den Arbeitnehmern die rechtliche Möglichkeit mitzureden, wenn es um betriebliche Belange geht.
\newline
Es gibt Betriebsräte auf betrieblicher Ebene, aber auch Gesamt- und Konzernbetriebsräte. In Verwaltungen und Behörden des öffentlichen Dienstes gibt es keine Betriebsräte sondern Personalräte, deren Rechte im Personalvertretungsgesetz des Bundes und der Länder geregelt sind.
\newline 
Betriebsräte werden alle vier Jahre im gleichen Zeitraum von März bis Mai gewählt. 
Während der Arbeitszeit finden die Wahlen statt. 
Wenn im Betrieb noch keine Interessenvertretung besteht, kann jederzeit eine Wahl durchgeführt werden.
\newline 
Um überhaupt einen Betriebsrat wählen zu können, muss ein selbständiger Betrieb vorliegen, der mindestens fünf ständig beschäftigte Arbeitnehmer unterhält (§ 1 BetrVG). \newline
Der Betriebsrat wird dann aus einer Reihe von mehreren Kandidaten von den wahlberechtigten Arbeitnehmern im Betrieb gewählt. Die Wahl darf von niemandem, auch nicht vom Arbeitgeber behindert oder verboten werden.
Die Initiative einen Betriebsrat zu gründen ist freiwillig und obliegt der Belegschaft, die in einer Betriebsversammlung den Wahlvorstand (§ 17 II BetrVG) wählt. \newline
Der genaue Ablauf der Wahl wird im Betriebsverfassungsgesetz geregelt (§§ 7 ff. BetrVG). \newline
Die Amtszeit des Betriebsrats beträgt grundsätzlich vier Jahre (§ 21 BetrVG). Nach vier Jahren finden dann in der Zeit vom 1. März bis 31. Mai Betriebsratswahlen statt (§ 13 I BetrVG).