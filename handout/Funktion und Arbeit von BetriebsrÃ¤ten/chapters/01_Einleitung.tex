Die Mitbestimmung im Betrieb ist in den letzten Jahrzehnten immer wichtiger geworden und deshalb auch gesetzlich geregelt. Sie bezeichnet die Teilnahme der Arbeitnehmer bzw. deren Vertreter an den Entscheidungsprozessen innerhalb eines Unternehmens. So wird denjenigen, deren Arbeits- und Lebensverhältnisse von den Entscheidungen anderer abhängig sind, eine Einfluss- und Gestaltungsmöglichkeit gewährleistet. 
\newline
Die betriebliche Mitbestimmung, wie sie durch den Betriebsrat repräsentiert wird, gibt den Arbeitnehmern die rechtliche Möglichkeit mitzureden, wenn es um betriebliche Belange geht.
\newline
Arbeitnehmer befinden sich durch das Arbeitsverhältnis in einer Abhängigkeit zu ihrem Arbeitgeber. Diese Abhängigkeit hat einen Einfluss nicht nur auf die Arbeitsgestaltung, sondern im weiteren Sinne auch auf die Lebensweise und Lebensverhältnisse der Arbeitnehmer. 
\newline
Zwar sind Arbeitgeber aufgrund der rechtlichen Besitzverhältnisse grundsätzlich frei in ihren unternehmerischen Entscheidungen, diese sind allerdings durch die rahmengebende Gesetzgebung zur Mitbestimmung im Betrieb begrenzt. Dies dient dem Schutz der Arbeitnehmer und findet beispielsweise Anwendung in folgenden Bereichen:
\begin{itemize}
	\item Mitbestimmung bei der Arbeitsgestaltung und Rahmenbedingungen durch Vorschlagsrecht
	\item Anspruch auf Aufklärung zur auszuübenden Tätigkeit und damit verbundener Verantwortung
	\item Einhaltung des Arbeitsschutzes und der Beurteilung von Gefährdungen
	\item Recht der Arbeitnehmer auf Einsicht in die Personalakte
	\item Recht der Vertreter der Arbeitnehmer zur Mitbestimmung bei unternehmerischen Entscheidungen, wie
	\begin{itemize}
		\item Zeiterfassung der Arbeitnehmer
		\item Kontroll- und Bewertungssysteme der Arbeitnehmer
		\item Personalplanung
		\item Sozialplan und Interessenausgleich bei unternehmerischen Umstrukturierungen
		\item Einführung von Incentives und anderen Anreizsystemen
		\item Auswahl von Mitarbeitern und deren Ausscheiden durch Kündigung
		\item Ausgestaltung von Betriebsvereinbarungen
	\end{itemize}
\end{itemize}