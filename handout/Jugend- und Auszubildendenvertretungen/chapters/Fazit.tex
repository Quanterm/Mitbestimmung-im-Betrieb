Neben den Betriebsrätinnen und -räten als Vertreterinnen und Vertreter aller Beschäftigten gibt es in vielen Betrieben auch eine JAV, die speziell die Interessen von Jugendlichen und Azubis vertritt. Die JAV unterstützt Azubis bei allen wichtigen Fragen rund um die Ausbildung. Wichtig ist, dass die JAV sehr eng mit uns zusammenarbeitet. Damit sie sich inhaltlich immer auf dem Laufenden hält, regelmäßig Schulungen bekommt und sich mit JAVen aus anderen Betrieben austauschen kann.
\newline 
Sie berät Jugendliche und Auszubildende in Fragen zu Arbeit und Ausbildung und achtet darauf, dass Gesetze und Tarifverträge im Betrieb eingehalten werden. Hierzu macht sie Druck für die Übernahme nach der Ausbildung und kümmert sich um die Gleichstellung von Frauen und Migranten im Unternehmen.
Es werden regelmäßig Sitzungen geben, wo alle anfallenden Probleme besprochen werden um eine Lösung zu finden.
\newline
Das Betriebsverfassungsgesetz regelt die Pflichten und Rechte der JAV. Eine Aufgabe der JAV: Sie überwacht, dass die Auszubildenden korrekt behandelt werden – und dass die Chefin oder der Chef die für sie geltenden Gesetze, Bestimmungen und Tarifverträge einhält. Die JAV hilft, wenn es etwa Probleme mit dem Meister gibt. Oder wenn die Qualität der Ausbildung nicht stimmt – etwa weil ein Azubi Kaffee kochen und putzen muss statt seinen Beruf zu erlernen.
\newline
Die JAV organisiert zudem Aktionen, etwa auf Betriebsversammlungen, mit denen Auszubildende und Jugendliche ihre Forderungen klarmachen. Nicht nur der Chefin oder dem Chef, sondern auch der Belegschaft. Denn schließlich brauchen sie auch die Unterstützung der älteren Beschäftigten, um ihre Ziele durchzusetzen. Außerdem setzt die JAV mit dem Betriebsrat durch, dass der Betrieb alle Azubis nach der Abschlussprüfung unbefristet übernimmt.