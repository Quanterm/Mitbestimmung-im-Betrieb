\subsection{Pflichten}
{
Die Jugend- und Auszubildendenvertretung arbeitet eng mit dem Betriebsrat bzw. der Personalvertretung zusammen. 
}
\newline
\begin{itemize}
	\item  
	Wahrnehmung der Belange der Auszubildenden
	\item
	Beantragung von Maßnahmen beim Betriebsrat oder der Personalvertretung (speziell zu Ausbildung, Übernahme, Gleichstellung von Männern und Frauen)
	\item
	Überwachung von Gesetzen, Vorschriften, Tarifverträgen usw.
	\item
	Anregungen der Auszubildenden an den Betriebs-/Personalrat
	\item
	\textbf{Kontrollpflicht:}
	\newline
	{
		Rechtsvorschriften, die Jugendliche und Auszubildende betreffen, müssen im Betrieb eingehalten werden. \newline 
		Darüber zu wachen gehört ebenso zu den Rechten wie zu den Pflichten einer JAV (§ 70 Abs. 1 Nr. 2 BetrVG). Unter diese Vorschriften fallen beispielsweise:
	}
	\begin{itemize}
		\item 
		
	\end{itemize}
	\item 
	\textbf{Geheimhaltungspflicht:}
	\newline
	{
		Alle Mitglieder und Ersatzmitglieder der JAV sind verpflichtet, Betriebs- oder Geschäftsgeheimnisse für sich zu behalten, sie also nicht zu offenbaren (§ 79 Abs. 2 BetrVG). Mit der Geheimhaltungspflicht ist dafür gesorgt, dass die JAV alle für ihre Arbeit relevanten Infos erhält – auch wenn sie brisant sind.
	 	\newline
	 	Darunter fallen aber nur Betriebs- und Geschäftsgeheimnisse, deren Offenbarung geschäftliche Nachteile gegenüber Dritten verursachen kann. Das sind Tatsachen, die vier Kriterien erfüllen: 
	 }
	\begin{itemize}
		\item 
		Sie stehen im Zusammenhang mit dem technischen Betrieb oder der wirtschaftlichen Betätigung des Unternehmens.
		\item
		Sie sind nur einem begrenzten betrieblichen Personenkreis bekannt.
		\item
		Sie sollen nach dem bekundeten Willen des Arbeitgebers geheim gehalten werden.
		\item
		Die Geheimhaltung ist für das Unternehmen wichtig.
	\end{itemize}
	Lohn- und Gehaltsdaten sind dagegen keine Betriebs- oder Geschäftsgeheimnisse.
	\newline 
	Die JAV ist erst geheimhaltungspflichtig, nachdem der Arbeitgeber den Sachverhalt ausdrücklich als geheimhaltungsbedürftig bezeichnet hat. Sogenannte „vertrauliche Angaben“ des Arbeitgebers unterliegen nicht der Schweigepflicht.
\end{itemize}

\subsection{Rechte}
{
Schutzvorschriften
Um eine Benachteiligung von Jugend- und Auszubildendenvertretern aufgrund ihres Amtes zu verhindern, hat der Gesetzgeber im Betriebsverfassungsgesetz und analog auch in den Personalvertretungsgesetzen diverse Schutzvorschriften für die Mitglieder einer JAV erlassen. So sind sie gemäß § 78 a BetrVG nach der Ausbildung in ein unbefristetes Arbeitsverhältnis zu übernehmen. Durch diese Verpflichtung zur Übernahme wird die Ämterkontinuität der Arbeitnehmervertretung gewährleistet und der Amtsinhaber somit vor negativen Folgen bei seiner Amtsausführung während des Berufsausbildungsverhältnisses geschützt.[2] Ausnahmen hiervor gibt es nur, wenn der Arbeitgeber nachweisen kann, dass er keinerlei Personalbedarf hat. Dies setzt voraus, dass kein anderer Auszubildender übernommen wird. Außerdem darf der Jugend- und Auszubildendenvertreter in seiner Abschlussprüfung nicht deutlich schlechter abgeschnitten haben, als der Durchschnitt seines Lehrjahres. Darüber hinaus genießen die Mitglieder der JAV denselben Kündigungsschutz wie die des Betriebsrates. Hat ein Arbeitgeber dennoch nicht vor, eine/*n Auszubildende/*n in der JAV, nach dem Ende des Berufsausbildungsverhältnisses auf unbestimmte Zeit zu übernehmen, so muss er das dem/r Auszubilden(en) schriftlich drei Monate vor Beendigung des Berufsausbildungsverhältnisses mitteilen (§ 78a Abs. 1 BetrVG). Dies ist ebenso gültig, wenn das Berufsausbildungsverhältnis vor Ablauf eines Jahres nach Beendigung der Amtszeit der JAV endet (§ 78a Abs. 3 BetrVG).
}
\begin{itemize}
	\item
	\textbf{Unterrichtungsrecht}
	\item 
	\textbf{Antragsrecht:}
	\newline 
	{
	Teilnahme an Betriebsratssitzungen und Verhandlungen mit dem Arbeitgeber. 
	\newline
		Beschäftigte, die sich an die JAV gewandt haben, muss sie während des gesamten Vorgangs über den aktuellen Stand informieren. \newline 
		\underline{Das darf sie tun über:}
	}
	\begin{itemize}
		\item 
		ein persönliches Gespräch
		\item
		ein Schreiben an die Betroffenen
		\item
		eine Jugend- und Auszubildendenversammlung 
		\item
		ein Infoblatt inkl. Bericht
		\item
		eine kreative Aktion
	\end{itemize}
	\item 
	\textbf{Anregungsrecht}
	\newline 
	{
		Alle Jugendlichen, Auszubildenden und dual Studierende sind berechtigt, sich während ihrer Arbeits- oder Ausbildungszeit mit Anregungen (§ 70 Abs. 1 Nr. 3 BetrVG) oder Beschwerden an die JAV zu wenden (§§ 84 und 85 BetrVG).
		\newline
		Die JAV ist verpflichtet, diese Anregungen anzunehmen. Sie können alle betrieblichen Fragen berühren, auch ohne speziell Jugendliche, Auszubildende oder dual Studierende zu betreffen. Auf der JAV-Sitzung muss sich die JAV mit den Anregungen befassen und prüfen, ob sie berechtigt sind.
		\newline
		Hält die JAV die Anregung für unberechtigt oder unrealistisch, muss sie darüber einen Beschluss fassen und die betroffene Person informieren. Dabei sollte die JAV nicht zu enge Grenzen setzen. Schließlich ist es ihre Aufgabe, Möglichkeiten zu betrieblicher Mitbestimmung zu eröffnen, nicht sie einzuschränken.
		\newline
		Stuft die JAV eine Anregung als berechtigt ein, muss sie beim Betriebsrat (BR) auf deren Umsetzung pochen. Dazu informiert sie den BR, nicht den Arbeitgeber. Der BR prüft die Angelegenheit unabhängig vom JAV-Beschluss und muss mit dem Arbeitgeber in Verhandlungen treten, soweit er die Anregung für berechtigt hält.
	}
	\item
	Teilnahme an erforderlichen Schulungen auf Kosten des Arbeitgebers
	\item
	Freistellungen von der arbeitsvertraglichen Pflicht zur Erledigung der Aufgaben als JAV-Mitglied ohne Minderung des Arbeitsentgeltes oder einem Verlust der eingebrachten Arbeitszeit
	\item 

\end{itemize}
