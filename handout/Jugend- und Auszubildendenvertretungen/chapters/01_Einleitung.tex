Die Jugend- und Auszubildendenvertretung (JAV) ist die Vertretung der Jugendlichen unter 18 Jahren und 
der zur Berufsausbildung Beschäftigten (Auszubildende, Praktikanten, Werkstudenten) in einem Betrieb oder einer Behörde in Deutschland.
Diese Personengruppe ist daher auch wahlberechtigt.
\newline
Eine Jugend- und Auszubildendenvertretung kann nur gewählt werden, wenn bereits ein Betriebsrat besteht. 
Eine Doppelmitgliedschaft in Betriebsrat und Jugend- und Auszubildendenvertretung ist im Betriebsverfassungsgesetz nicht vorgesehen. 